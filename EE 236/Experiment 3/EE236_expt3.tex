\documentclass[12pt]{article}
\usepackage{graphicx}
  
%
% Title.
\title{EE236: Experiment 3\\
Hall Effect Sensors}

% Author
\author{Name of student, Roll. no.}

% begin the document.
\begin{document}

% make a title page.
\maketitle

\section{Aim of the experiment}

(3 marks) In your own words, describe the aim of the experiment. There are three parts to this experiment, list down the aims of these parts.

\section{Methods}

In your own words, describe how you set out to realize the goal of the experiment. A few lines of brief overview of your approach are expected here. Do not list your observations here.

\subsection{Part 1}

(2 marks) Enter methods here. Circuit diagram is not mandatory. If you made any changes to the circuit, note them down and justify your choice.

\subsection{Part 2}

(2 marks) Enter methods here. Circuit diagram is not mandatory. If you made any changes to the circuit, note them down and justify your choice. 

\subsection{Part 3}

(2 marks) Enter methods here. Circuit diagram is not mandatory. If you made any changes to the circuit, note them down and justify your choice.

\section{Observations}

\subsection{Part 1}

(2 marks) Neatly tabulate your results. Specify experimental conditions. Note down observations, not inferences. If you faced technical challenges in performing the experiment, list them out in technically correct language here.

Add distance vs voltage plot (made in gnuplot or origin). What observations can you make in this plot?

There is no penalty if you couldn't finish due to technical challenges, but we will evaluate you on basis of how you report and debugged the problem.

\subsection{Part 2}

(2 marks) Neatly tabulate your results. Specify experimental conditions. Note down observations, not inferences. If you faced technical challenges in performing the experiment, list them out in technically correct language here.

Add current vs voltage plot (made in gnuplot or origin). What observations can you make in this plot?

There is no penalty if you couldn't finish due to technical challenges, but we will evaluate you on basis of how you report and debugged the problem.

\subsection{Part 3}

(2 marks) Specify experimental conditions. Note down observations, not inferences. If you faced technical challenges in performing the experiment, list them out in technically correct language here.

Add screenshots of the oscilloscope screen for both sensor outputs and add captions to the figures. Answer the questions asked in points 5 and 6 on page 14 of the lab handout slides.

There is no penalty if you couldn't finish due to technical challenges, but we will evaluate you on basis of how you report and debugged the problem.

\section{Simulation results}

(2 marks) Paste your ngspice code, and relevant plots, with suitable captions for all figures. Comment on your observations.

\section{Inference}

\subsection{Part 1}

(2 marks) Note down inferences drawn from observations in part 1. 

\subsection{Part 2}

(2 marks) Note down inferences drawn from observations in part 2.

\subsection{Part 3}

(2 marks) Note down inferences drawn from observations in part 3. Compare with simulation results, if applicable.

\section{Learning objectives}

(1 mark if you have written something, 0 if you don't write anything. This is mainly for your reference later in your degree program, and for your TAs and instructors to help you if needed.) According to you, what were the learning objectives in this experiment? Were they satisfactorily fulfilled?

\section{Quick feedback}

(0.5 marks each if you have written something, 0 if you don't write anything. This is mainly for your reference later in your degree program, and for your TAs and instructors to help you if needed.)

\subsection{What about this experiment did you find helpful?}

\subsection{What about this experiment is still unclear?}

\end{document}
