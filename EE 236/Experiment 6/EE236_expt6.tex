\documentclass[12pt]{article}
\usepackage{graphicx}

%
% Title.
\title{EE236: Experiment 6\\
Minority carrier lifetime measurement}

% Author
\author{Name of student, Roll. no.}

% begin the document.
\begin{document}

% make a title page.
\maketitle

\section{Aim of the experiment}

(3 marks) In your own words, describe the aims of both parts of the experiment.

\section{Methods}

In your own words, describe how you set out to realize the goal of the experiment for each of the parts. A few lines of brief overview of your approach are expected here. Do not list your observations here.

\subsection{Part 1}

(4 marks) Enter methods here. \textbf{Circuit diagram must be made in xcircuit - 2 marks reserved for circuit diagram}. If you made any changes to the circuit, note them down and justify your choice. If you need any equations to justify your methods, note them down too.

\subsection{Part 2}

(4 marks) Enter methods here. \textbf{Circuit diagram must be made in xcircuit - 2 marks reserved for circuit diagram}. If you made any changes to the circuit, note them down and justify your choice. If you need any equations to justify your methods, note them down too.

\section{Observations}

\subsection{Part 1}

(total 8 marks) Neatly tabulate your results and draw plots (4 marks). Specify experimental conditions. Note down observations, not inferences. If you faced technical challenges in performing the experiment, list them out in technically correct language here. Add neatly captioned oscilloscope screenshots (4 marks).

There is no penalty if you couldn't finish due to technical challenges, but we will evaluate you on basis of how you report and debugged the problem.

\subsection{Part 2}

(total 8 marks) Note down observations, not inferences. Add all screenshots you recorded (4 marks). If you faced technical challenges in performing the experiment, list them out in technically correct language here. Tabulate all your data recordings and draw neat and legible plots (4 marks).

There is no penalty if you couldn't finish due to technical challenges, but we will evaluate you on basis of how you report and debugged the problem.

\section{Inference}

\subsection{Part 1}

(4 marks) Note down inferences drawn from observations in part 1.

\subsection{Part 2}

(4 marks) Note down inferences drawn from observations in part 2.

\section{Reflection questions}

(ungraded)

1. In part-2, when the square wave siwtches from positive half cycle to negative half cycle, is the diode in forward bias or reverse bias during the storage time duration? The voltage applied qualifies the diode as reverse biased, but there is a large current flowing due to excess carriers!! Make sure you ask this question during your EE207 class when they teach you carrier lifetime.\\
Ans.\\

2. Based on what you observed during this lab, what do you think is the equivalent circuit of a diode (an equivalent circuit is a simplified circuit that presents the same response as the diode to applied voltage, and is represented in terms of resistors, capacitors, inductors, voltage/current sources, and any other elementary circuit components).\\
Ans.\\

3. You used Schottky diode in the last lab. How would you expect the response to be if you used Schottky diode in this experiment in part-2?\\
Ans.\\

\end{document}
