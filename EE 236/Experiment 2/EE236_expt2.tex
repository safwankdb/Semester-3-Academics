\documentclass[12pt]{article}
\usepackage{graphicx}

%
% Title.
\title{EE236: Experiment 2\\
Simple op-amp circuits}

% Author
\author{Mohd Safwan, 17D070047}

% begin the document.
\begin{document}

% make a title page.
\maketitle

\section{Aim of the experiment}

To study fequency response and phase of an RC-Circuit.
\newline
To make different op-amp circuits.

\section{Methods}

In your own words, describe how you set out to realize the goal of the experiment. About 2 paragraphs of brief overview of your approach are expected here. Do not list your observations here.

\subsection{Part 1}

(2 marks) Enter methods here. Draw the circuit diagram in xcircuit the way you connected in your setup.

\subsection{Part 2}

(2 marks) Enter methods here. Draw the circuit diagram(s) in xcircuit the way you connected in your setup.

\section{Observations}

\subsection{Part 1}

(2 marks) Neatly tabulate your results. Specify experimental conditions. Note down observations, not inferences. If you faced technical challenges in performing the experiment, list them out in technically correct language here.

Add relevant plots (made in gnuplot or origin) and screenshots.

There is no penalty if you couldn't finish due to technical challenges, but we will evaluate you on basis of how you report and debugged the problem.

\subsection{Part 2}

(2 marks) Neatly tabulate your results. Specify experimental conditions. Note down observations, not inferences. If you faced technical challenges in performing the experiment, list them out in technically correct language here.

Add relevant plots (made in gnuplot or origin) and screenshots.

There is no penalty if you couldn't finish due to technical challenges, but we will evaluate you on basis of how you report and debugged the problem.

\section{Simulation results}

(5 marks) Paste your ngspice code, and relevant plots, with suitable captions for all figures. Comment on your observations.

\section{Inference}

\subsection{Part 1}

(2 marks) Note down inferences drawn from observations in part 1. Compare with simulation results, if applicable.

\subsection{Part 2}

(2 marks) Note down inferences drawn from observations in part 2. Compare with simulation results, if applicable.

\section{Questions for reflection}

(4 marks)

1. What is the input impedance of a non-inverting amplifier? What is the input impedance of an inverting amplifier?\\
Ans. Enter answer here.
\\\\
2. If I have to amplify the output of a sensor that has a capacitive transducer at the output, which amplifier (inverting or non-inverting) is better from an impedance matching perspective? Justify your answer, using technically correct language that befits a good electrical engineer. If some terms are not clear, refer a good textbook.\\
Ans. Enter answer here.
\\\\
3. In the op-amp adder circuit, if the non-inverting terminal is connected to $+2V$ instead of ground, how will the output change?\\
Ans. Enter answer here.
\\\\
4. You noticed that the op-amp IC did not have any pin that was supposed to be connected to ground. Is `ground' an abstract concept in case of an op-amp? Could one define any arbitrary voltage as ground? Try to answer this question intuitively.
\end{document}
