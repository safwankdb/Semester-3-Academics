\documentclass[12pt]{article}
\usepackage{graphicx}

%
% Title.
\title{EE236: Experiment 5\\
Suggest a suitable title}

% Author
\author{Name of student, Roll. no.}

% begin the document.
\begin{document}

% make a title page.
\maketitle

\section{Aim of the experiment}

(3 marks) In your own words, describe the aim of the experiment. There are three parts to this experiment, list down the aims of these parts.

\section{Methods}

In your own words, describe how you set out to realize the goal of the experiment for each of the parts. A few lines of brief overview of your approach are expected here. Do not list your observations here.

\subsection{Part 1}

(2 marks) Enter methods here. If you made any changes to the circuit, note them down and justify your choice.

\subsection{Part 2}

(2 marks) Enter methods here. If you made any changes to the circuit, note them down and justify your choice.

\section{Observations}

\subsection{Part 1}

(2 marks) Neatly tabulate your results. Specify experimental conditions. Note down observations, not inferences. If you faced technical challenges in performing the experiment, list them out in technically correct language here.

Draw plot1, plot2 and plot of $V_\gamma$ v/s $E_g$ as mentioned in the handout. Draw plots of all diodes overlaid in same image, instead of drawing separate images for each diode. What observations can you make in these plots? (e.g. slope of curve, nonlinearity etc.)

There is no penalty if you couldn't finish due to technical challenges, but we will evaluate you on basis of how you report and debugged the problem.

\subsection{Part 2}

(2 marks) Note down observations, not inferences. Add all screenshots you recorded, for various input voltage amplitudes. If you faced technical challenges in performing the experiment, list them out in technically correct language here.

There is no penalty if you couldn't finish due to technical challenges, but we will evaluate you on basis of how you report and debugged the problem.

\section{Inference}

\subsection{Part 1}

(2 marks) Note down inferences drawn from observations in part 1. This includes the analysis and interpretation as mentioned in the handout.

\subsection{Part 2}

(2 marks) Note down inferences drawn from observations in part 2.

\section{Advanced component}

If you finished the advanced component circuit, draw the circuit diagram you used, and justify the choice of components. If you took a photograph of the circuit in action, add it here. State your observations. This section is not graded, but performing this part satisfactorily qualifies you for `AA' grade, as stated in course rules.

\section{Reflection questions}

(ungraded)

1. In the rectifier circuits in part-2, what is the main advantage of using a Schottky diode, as compared to a silicon pn junction?\\
Ans.\\

2. In heart rate monitor lab, you used a phototransistor as a detector. Do you think an LED can be used as a photodetector as well? Justify your answer.\\
Ans.\\

3. Without conducting any I/V measurements, did you find any other way in which you can distinguish the different diode types? Suggest how this may be achieved.\\
Ans.\\

\end{document}
